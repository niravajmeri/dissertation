\chapter{\frameworkA: Surveys}
\label{app:arnor-survey}

%We conducted a pre-participation survey to form two balanced groups of participants. 

\section{Pre-participation Survey}
\label{appsec:presurvey}

\begin{enumerate}
\item How long (in years) is your academic programming experience? Count a semester as six months or less depending on your programming effort that semester. An approximation is fine.
\begin{itemize}
\item[$\circ$] $<$1
\item[$\circ$] 1--2
\item[$\circ$] 2--6
\item[$\circ$] 6$+$
\end{itemize}

\item How long (in years) is your industry programming experience? Count full-time experience and approximate part-time experience.
\begin{itemize}
\item[$\circ$] $<$1
\item[$\circ$] 1--2
\item[$\circ$] 2--6
\item[$\circ$] 6$+$
\end{itemize}

\item How familiar are you with conceptual modeling? Conceptual modeling includes requirements engineering, system specification, architectural design, and so on.
\begin{itemize}
\item[$\circ$] Not familiar
\item[$\circ$] Familiar with the concepts, but no practical experience
\item[$\circ$] Familiar and some practical experience from an earlier academic project
\item[$\circ$] Familiar and some practical experience from the industry
\item[$\circ$] Expert (e.g., system architect, requirements engineer, researcher in this area, etc.)
\item[$\circ$] Other
\end{itemize}

\end{enumerate}

%\section{Time and Effort Survey-- Xipho Group}
%
%Answer this survey after each work session. It is best to answer the
%survey right after a work session. A work session is typically a
%sitting.
%
%Please be as accurate as possible. If you make a mistake such as
%submitting a survey twice or incorrectly reporting something, please
%email the investigator so that we can fix the mistake.
%
%\begin{enumerate}
%
%\item How long was this work session? In hours:minutes (e.g., 03:30). Please exclude interruptions from the session duration. If an interruption is long enough, it might be better to treat a session as two.
%\item How do you rate the difficulty of the work your performed in this session? Note that session duration and difficulty are not necessarily related.
%Answer on a numeric scale 1--7 where 1 is \emph{very easy}, and 7 is \emph{very difficult}.
%%Very Easy (1) -- Very Difficult (7)
%
%\item What did you do in this work session? 
%
%\begin{itemize}
%\item[$\circ$] Read project description
%\item[$\circ$] Learning Xipho
%\item[$\circ$] Learning Lucidchart
%\item[$\circ$] Reading other material (specify in comments)
%\item[$\circ$] Modeling
%\item[$\circ$] Documentation (specify in comments)
%\item[$\circ$] Other
%\end{itemize}
%
%\item What is an approximate breakdown of this work session? For example, [reading project description (10\%) + coding (65\%) + documentation (15\%)]. You can breakdown however you want (not necessary to breakdown three way like in the example). Also, briefly describe what you did in this session.
%
%\item Any additional comments?
%\item Anything else you may want to mention about this work session.
%
%\end{enumerate}

\section{Time and Effort Survey}
\label{appsec:effortsurvey}

Answer this survey after each work session. It is best to answer the
survey right after a work session. A work session is typically a
sitting.

Please be as accurate as possible. If you make a mistake such as
submitting a survey twice or incorrectly reporting something, please
email the investigator so that we can fix the mistake.

\begin{enumerate}

\item How long was this work session? In hours:minutes (e.g., 03:30). Please exclude interruptions from the session duration. If an interruption is long enough, it might be better to treat a session as two.
\item How do you rate the difficulty of the work your performed in this session? Note that session duration and difficulty are not necessarily related.
Answer on a numeric scale 1--7 where 1 is \emph{very easy}, and 7 is \emph{very difficult}. 
%Very Easy (1) -- Very Difficult (7)

\item What did you do in this work session? 

\begin{itemize}
\item[$\square$] Read project description
\item[$\square$] Learning and understanding methodology to develop socially-aware application
\item[$\square$] Learning Lucidchart
\item[$\square$] Reading other material (specify in comments)
\item[$\square$] Modeling
\item[$\square$] Workout Social Benefit Function (specify in comments)
\item[$\square$] Implementation (specify in comments)
\item[$\square$] Documentation (specify in comments)
\item[$\square$] Other
\end{itemize}

\item What is an approximate breakdown of this work session? For example, [reading project description (10\%) + coding (65\%) + documentation (15\%)]. You can breakdown however you want (not necessary to breakdown three way like in the example). Also, briefly describe what you did in this session.

\item Any additional comments?
\item Anything else you may want to mention about this work session.

\end{enumerate}

\section{Post Survey}
\label{appsec:postsurvey}

\subsection*{Overall Time and Difficulty}

\begin{enumerate}
\item Give an estimate of the overall time you spent in hours:minutes (e.g., 03:30) to understand the project requirements and to prepare the requirement specification (models)? 


\item Give an estimate of the overall time you spent in hours:minutes (e.g., 03:30) to design the social benefit function? 


\item Give an estimate of the overall time you spent in hours:minutes (e.g., 03:30) to implement the project? 


\item Give an estimate of the overall time you spent in hours:minutes (e.g., 03:30) to test the project? 


\item Give an estimate of the overall time you spent in hours:minutes (e.g., 03:30) to document the project? 



\item How easy were the following phases? 
Answer on a numeric scale 1--7 where 1 is \emph{very easy}, and 7 is \emph{very difficult}.
%1 (Very Easy)	2	3	4	5	6	7 (Very Difficult)

\begin{itemize}
\item Understanding Requirements	
\item Preparing requirement specification (models)	
\item Implementation	
\item Testing	
\item Documentation	
\end{itemize}

\item In what aspects do you think the application violates your privacy?

\end{enumerate}

\subsection*{Usability}
Consider yourself as a user of your application.

\begin{enumerate}
\item List all actions that you can perform using the application.


\item To what extent do you think the application preserves your privacy? 
Answer on a numeric scale 1--7 where 1 is \emph{application is privacy-preserving}, and 7 is \emph{application is privacy violating}.

\item How usable is your application considering the following aspects? 
Answer on a numeric scale 1--7 where 1 is \emph{very easy}, and 7 is \emph{very difficult}.
%%1 (Very Easy)	2	3	4	5	6	7 (Very Difficult)
\begin{itemize}
\item How easy is it to accomplish the actions the first time you use the application?	
\item Once you have learned the application, how quickly can you perform the actions?	
\item When you return to the application after a period of not using it, how easily can you re-establish proficiency?	
\item How many errors do you make when you use the application? (1: not many, 7: quite a lot)	
\item How severe are the errors? (1: not very severe, 7: very severe)	
\item How easily can it recover from the errors? (1: quickly, 7: takes a long time)	
\item How pleasant is it to use the application? (1: very pleasant, 7: not at all pleasant)	
\item How easy is it to accomplish the actions the first time you use the application?	
\item Once you have learned the application, how quickly can you perform the actions?	
\item When you return to the application after a period of not using it, how easily can you re-establish proficiency?	
\item How many errors do you make when you use the application? (1: not many, 7: quite a lot)	
\item How severe are the errors? (1: not very severe, 7: very severe)	
\item How easily can it recover from the errors? (1: quickly, 7: takes a long time)	
\item How pleasant is it to use the application? (1: very pleasant, 7: not at all pleasant)	
\end{itemize}
\end{enumerate}

\subsection*{Methodology and Project Development}
Understanding requirements, implementation, testing and documentation

\begin{enumerate}

\item How clear were (or are) the requirements of this project for you? 
Answer on a numeric scale 1--7 where 1 is \emph{very easy}, and 7 is \emph{very difficult}.
%1 (Very Easy)	2	3	4	5	6	7 (Very Difficult)
\begin{itemize}
\item When you started the project	
\item Now, at the end of the project	
\end{itemize}

\item To what extent did the methodology help you in the following aspects of the project? 
Answer on a numeric scale 1--7 where 1 is \emph{didn't help me at all}, and 7 is \emph{helped me a lot}.
%1 (Didn't help me at all)	2	3	4	5	6	7 (Helped me a lot)
\begin{itemize}
\item Understanding the project requirements	
\item Implementing the project requirements	
\item Testing	
\item Documentation	
\end{itemize}

\item How easy is it to understand your implementation for someone else? 
Answer on a numeric scale 1--7 where 1 is \emph{very easy}, and 7 is \emph{very difficult}.
%1 (Very Easy)	2	3	4	5	6	7 (Very Difficult)
\begin{itemize}
\item Who knows the methodology	
\item Who does not know the methodology	
\end{itemize}

\item How easy is it to understand your documentation for someone else? 
Answer on a numeric scale 1--7 where 1 is \emph{very easy}, and 7 is \emph{very difficult}.
%1 (Very Easy)	2	3	4	5	6	7 (Very Difficult)
\begin{itemize}
\item Who knows the methodology	
\item Who does not know the methodology	
\end{itemize}

\item How easy was (or is) it to identify or incorporate actors? 
Answer on a numeric scale 1--7 where 1 is \emph{very easy}, and 7 is \emph{very difficult}.
%1 (Very Easy)	2	3	4	5	6	7 (Very Difficult)
\begin{itemize}
\item (For you to) Identify all actors in the scenario description?	
\item (For someone else to) Identify all actors in your implementation?	
\item (For you to) Incorporate a new actor in your implementation?	
\end{itemize}

\item How easy was (or is) it to identify or incorporate actions? 
Answer on a numeric scale 1--7 where 1 is \emph{very easy}, and 7 is \emph{very difficult}.
%1 (Very Easy)	2	3	4	5	6	7 (Very Difficult)
\begin{itemize}
\item (For you to) Identify all actions in the scenario description?	
\item (For someone else to) Identify all actions in your implementation?	
\item (For you to) Incorporate a new action in your implementation?	
\end{itemize}

\item How easy was (or is) it to identify or incorporate contexts in the actions? 
Answer on a numeric scale 1--7 where 1 is \emph{very easy}, and 7 is \emph{very difficult}.
%1 (Very Easy)	2	3	4	5	6	7 (Very Difficult)
\begin{itemize}
\item (For you to) Identify contexts in the scenario description?	
\item (For someone else to) Identify contexts in your implementation?	
\item (For you to) Incorporate a new context in your implementation?	
\end{itemize}

\item How easy was (or is) it to identify or incorporate norms? 
Answer on a numeric scale 1--7 where 1 is \emph{very easy}, and 7 is \emph{very difficult}.
%1 (Very Easy)	2	3	4	5	6	7 (Very Difficult)
\begin{itemize}
\item (For you to) Identify all norms in the scenario description?	
\item (For someone else to) Identify all norms in your implementation?	
\item (For you to) Incorporate a new norms in your implementation?	
\end{itemize}

\item How easy was (or is) it to identify norm conflicts and inconsistencies? 
Answer on a numeric scale 1--7 where 1 is \emph{very easy}, and 7 is \emph{very difficult}.
%1 (Very Easy)	2	3	4	5	6	7 (Very Difficult)
\begin{itemize}
\item (For you to) Identify norm conflicts in the scenario description?	
\item (For someone else to) Identify conflict resolution in your implementation	
\end{itemize}

\item In your implementation, how did you resolve conflicts between norms?  If you answered this in the report, please paste it here.

\item Do you think the methodology missed some crucial step that could have helped in understanding the project requirements, implementation, testing, or documentation?

\item Any other comments or feedback.
\end{enumerate}