%% ------------------------------ Abstract ---------------------------------- %%
% \begin{abstract}

% A socially intelligent personal agent
% understands and helps its user navigate the norms governing the user's
% interaction in a society. This research addresses the research question
% of how can we engineer social intelligence in a personal agent such 
% that it selects ethically appropriate actions and delivers a pleasant and 
% privacy-preserving social experience. Addressing this research question,
% we develop multiagent system techniques to engineer such socially
% intelligent and ethical personal agents.

% This research develops 
% \begin{enuminline}
% \item \frameworkA, a software engineering method to model social
% intelligence in privacy-aware personal agents, 
% \item \frameworkB, an
% approach that enables personal agents to reason about shared contexts,
% and learn contextually relevant social norms that preserve privacy, and
% \item \frameworkAinur, a decision-making framework to design agents that can reason about values, and act ethically. 
% \end{enuminline}


% \frameworkA goes beyond traditional software engineering methods to engineer personal
% agents by systematically capturing interactions that influence social
% experience.

% A personal agent may deviate from norms in certain contexts.
% \frameworkB 
% \begin{enuminline}
% \item enables personal agents deviating from norms to share
% deviation contexts with other agents in the agent society, and 
% \item provides personal agents the ability to reason about shared contexts.
% \end{enuminline}

% Privacy, values, and ethics are closely intertwined. 
% Preserving privacy presumes understanding of human values 
% and acting ethically. 
% \frameworkAinur equips a personal agent with an understanding of 
% values such as pleasure, privacy, recognition, and security, its actions promote or demote. This understanding of values 
% helps personal agents to select ethically appropriate actions especially 
% in scenarios where either the norms conflict or the value preferences of the users are not aligned.

% We claim that 
% \begin{enuminline}
% \item \frameworkA supports developers
% in engineering personal agents, and 
% \item personal agents engineered using
% \frameworkA provide a better privacy-preserving social experience than agents
% engineered using a traditional software engineering method.
% \end{enuminline}
% We evaluate \frameworkA via
% a developer study, and a set of simulation experiments, and measure the
% social experience via metrics of norm compliance and sanction
% proportion.

% We make two claims about the impact of context sharing and reasoning in
% \frameworkB. First, the ability to reason about deviation contexts helps
% a personal agent to accurately infer contextually relevant norms and to act in a 
% privacy respecting manner.
% Second, by acting according to such contextually relevant norms, a personal agent yields higher goal satisfaction to its users than an agent that does not reason about
% shared contexts. We demonstrate these claims via social simulations
% involving agent societies of varying sizes and diverse characteristics
% reflecting pragmatic, considerate, and selfish agents.

% We claim that the ability in a personal agent, designed using \frameworkAinur, to  understand its human users' values helps the agent select ethical actions. 
% We empirically evaluate \frameworkAinur via multiple simulation 
% experiments. We find that agents developed using \frameworkAinur produce ethical actions that exhibit the Rawlsian property of fairness and yield a pleasant social experience to its users.

% \end{abstract}

\begin{abstract}
A socially intelligent personal agent understands and helps its user navigate the norms governing the user's interaction in a society. This research seeks to advance the science of privacy by tackling nuanced notions of privacy in personal agents. It addresses the research question of how we can engineer social intelligence in a personal agent such that it selects ethically appropriate actions and delivers a pleasant and privacy-preserving social experience. Addressing this research question, we develop multiagent system techniques to engineer such socially intelligent and ethical personal agents.

This research develops (1) \frameworkA, a software engineering method to engineer privacy-aware personal agents by modeling social intelligence via norms, (2) \frameworkB, an approach that enables personal agents to reason about shared contexts and infer contextually relevant social norms that preserve privacy, and (3) \frameworkAinur, a decision-making framework to design personal agents that can reason about values and act ethically. 

\frameworkA goes beyond traditional software engineering methods to engineer personal agents by systematically capturing interactions that influence social experience. We claim that (1) \frameworkA supports developers in engineering intelligent personal agents, and (2) personal agents engineered using \frameworkA provide a better privacy-preserving social experience than agents engineered using a traditional software engineering method. We evaluate \frameworkA via a developer study and a set of simulation experiments and measure the social experience via metrics of norm compliance and sanction proportion.

A personal agent may deviate from norms in certain contexts. \frameworkB (1) enables personal agents deviating from norms to share deviation contexts with other agents in the agent society and (2) provides personal agents the ability to reason about shared contexts. We make two claims about the impact of context sharing and reasoning in \frameworkB. First, the ability to reason about deviation contexts helps a personal agent to accurately infer contextually relevant norms and to act in a privacy-respecting manner. Second, by acting according to such contextually relevant norms, a personal agent yields higher goal satisfaction to its users than an agent that does not reason about shared contexts. We demonstrate these claims via social simulations involving agent societies of varying sizes and diverse characteristics reflecting pragmatic, considerate, and selfish agents.

Privacy, values, and ethics are closely intertwined. Preserving privacy presumes  understanding human values and acting ethically. \frameworkAinur equips a personal agent with an understanding of values such as pleasure, privacy, recognition, and security, promoted or demoted by the agent's actions. This understanding of values helps personal agents to select ethically appropriate actions especially in scenarios where either the norms conflict or the value preferences of the users are not aligned.  We empirically evaluate \frameworkAinur via multiple simulation experiments. We find that agents developed using \frameworkAinur produce ethical actions that exhibit the Rawlsian property of fairness and yield a pleasant social experience to the agents' users.
\end{abstract}


%% ---------------------------- Copyright page ------------------------------ %%
%% Comment the next line if you don't want the copyright page included.
\makecopyrightpage

%% -------------------------------- Title page ------------------------------ %%
\maketitlepage

%% -------------------------------- Dedication ------------------------------ %%
\begin{dedication}
  \centering 

%  \emph{To my grandparents, my parents, my sister, and my wife.}
%
%  \emph{Without their love, support, and encouragement, this would not have been possible.}
\end{dedication}

%% -------------------------------- Biography ------------------------------- %%
\begin{biography}
% The author was born in a small town \ldots
\end{biography}

%% ----------------------------- Acknowledgements --------------------------- %%
\begin{acknowledgements}
% I would like to thank my advisor for his help.
\end{acknowledgements}


\thesistableofcontents

\thesislistoftables

\thesislistoffigures
