%------------------------------%
\chapter{Conclusions and Directions}
\label{chap:conclusions}
%------------------------------%

\section{Conclusions}

\begin{itemize}
\item Seeking to advance the science of privacy by tackling nuanced
notions of privacy (understood as an ethical value) in personal agents
\item Contributions:
\begin{description}[nosep]
\item[Modeling social intelligence:] \frameworkA, a software engineering method to engineer privacy-aware personal agents 
(\fbf{Fairness}; \fbf{Accountability})
\item[Understanding social context:] \frameworkB, an approach that enables personal agents to infer contextually relevant social norms that preserve privacy (\fbf{Accountability}; \fbf{Transparency})
\item[Understanding value preferences:] \frameworkAinur, a decision-making framework to design personal agents that can reason about values and act ethically (\fbf{Fairness}; \fbf{Ethics})
\end{description}
\end{itemize}

\section{Possible Directions for Future Dissertations}

\begin{itemize}
    \item Artificial Intelligence
    \begin{description}[nosep]
        \item[Social reality:] White lies and affect in personal agents (building on IJCAI 2018 and Trust 2014 works)
        \item[Formal specification:] Argumentation and value-based reasoning (building on Computer 2017 and IJCAI 2016 works)
    \end{description}
    
    \item Software Engineering
    
    \begin{description}[nosep]
        \item[Creativity:] CrowdRE for privacy requirements (building on RE 2016 and RE 2018 works)
        \item[Social reality:] RE for ethical systems (building on AAMAS 2017)
    \end{description}
    
    \item Privacy
    
    \begin{description}[nosep]
        \item[Social reality:] Middleware based on \frameworkAinur as a privacy-enhancing technology to support ethical decision-making 
        \item[Social reality:] Usable privacy and ethics
        
    \end{description}
\end{itemize}