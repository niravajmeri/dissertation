%------------------------------%
\chapter{Conclusions and Directions}
\label{chap:conclusions}
%------------------------------%

\section{Conclusions}

We tackle nuanced notion of privacy, understood as an ethical value, from a sociotechnical viewpoint. 
Specifically we address the challenges of understanding social reality---understanding social expectations, social context, values, and ethics.

\frameworkA assists software developers to engineer personal agents by capturing stakeholders' social expectations, goals, and plans, and how these influence each other.  
Social expectation modeling via social norms in \frameworkA enables capturing accountability,
and social experience modeling in \frameworkA helps incorporating fairness in decision-making.

\frameworkB enables personal agents to understand social context, and infer contextually relevant social norms that respect stakeholders' privacy. 
Revealing and reasoning about social contexts to infer contextually relevant norms yields both transparency and accountability.

\frameworkAinur provides personal agents with a decision-making ability to understand and reason about stakeholders' value preferences, 
and accordingly select ethically appropriate actions, thereby yields fairness.  

%\begin{itemize}
%\item Seeking to advance the science of privacy by tackling nuanced
%notions of privacy (understood as an ethical value) in personal agents
%\item Contributions:
%\begin{description}[nosep]
%\item[Modeling social intelligence:] \frameworkA, a software engineering method to engineer privacy-aware personal agents 
%(\fbf{Fairness}; \fbf{Accountability})
%\item[Understanding social context:] \frameworkB, an approach that enables personal agents to infer contextually relevant social norms that preserve privacy (\fbf{Accountability}; \fbf{Transparency})
%\item[Understanding value preferences:] \frameworkAinur, a decision-making framework to design personal agents that can reason about values and act ethically (\fbf{Fairness}; \fbf{Ethics})
%\end{description}
%\end{itemize}

\section{Possible Directions for Future Dissertations}
Future dissertations can be pursued in three dimensions---artificial intelligence, software engineering, and privacy.

\paragraph*{Artificial Intelligence}
In the artificial intelligence dimension, support to model white lies when revealing context and incorporating affect in personal agents to promote social cohesion and privacy is a promising future direction \cite{IJCAI-18:Poros}.
Adopting argumentation and value-based reasoning to model and to infer preferences among values is another future direction \cite{Ajmeri-IJCAI16-Coco}.

\paragraph*{Software Engineering}
CrowdRE is a promising avenue for engaging crowd in human-intensive tasks such as capturing requirements for a personal agent like the Ringer SIPA described in Chapters~\ref{chap:arnor} and \ref{chap:poros}, and the Pichu SIPA described in Chapter~\ref{chap:ainur}. 
A first direction in the software engineering dimension is developing new techniques that incorporate creativity in the CrowdRE process to capture privacy requirements from stakeholders \cite{Murukannaiah-RE16-Creative}. 
A second direction is to design a requirements engineering approach to assist software developers in developing ethical social applications \cite{Ajmeri-AAMAS17-Arnor,Ajmeri-IC18-Ethical}. 


\paragraph*{Privacy}

A first direction in the privacy dimension is developing a privacy-enhancing middleware based on \frameworkB and \frameworkAinur to support ethical decisision-making in social applications \cite{Ajmeri-hotsos18-Ethics,Murukannaiah-TOSEM15-Platys}.
A second direction is to develop recommendation systems around these frameworks to tackle usability issues in privacy, security, and ethics.
 

%\begin{itemize}
%    \item Artificial Intelligence
%    \begin{description}[nosep]
%        \item[Social reality:] White lies and affect in personal agents (building on IJCAI 2018 and Trust 2014 works)
%        \item[Formal specification:] Argumentation and value-based reasoning (building on Computer 2017 and IJCAI 2016 works)
%    \end{description}
%    
%    \item Software Engineering
%    
%    \begin{description}[nosep]
%        \item[Creativity:] CrowdRE for privacy requirements (building on RE 2016 and RE 2018 works)
%        \item[Social reality:] RE for ethical systems (building on AAMAS 2017)
%    \end{description}
%    
%    \item Privacy
%    
%    \begin{description}[nosep]
%        \item[Social reality:] Middleware based on \frameworkAinur as a privacy-enhancing technology to support ethical decision-making 
%        \item[Social reality:] Usable privacy and ethics
%        
%    \end{description}
%\end{itemize}
